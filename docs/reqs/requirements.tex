\author{}
\title{SE 3354.003 - Software Engineering1}
\documentclass[12pt,letterpaper,oneside]{article}
\usepackage{requirements}

% Metadata
\usepackage[pdftex,
	pdftitle={SE 3354.003 Project Requirements},
	pdfsubject={Software Engineering},
	pdfproducer={LaTeX with hyperref},
	pdfcreator={pdflatex}]{hyperref}

% Headers and foots
\pagestyle{fancy}
\fancyhf{}
\lhead{SE 3354.003}
\chead{Project Requirements}
\rhead{MinuteMath}

\begin{document}
%===============================================================================
\begin{enumerate}
	\item
		The application shall contain games to reinforce learning of mathematical concepts
	\item
		% UI Reqs
		The application shall implement a graphical user interface (UI)
		\begin{enumerate}
			\item
				% Main menu
				The appplication shall show a main menu on initial launch
				\begin{enumerate}
					\item
						The main menu shall contain buttons to user profile, games, and tutorials
				\end{enumerate}
			\item
				% Tutorials
				The UI shall contain a tutorial section
				\begin{enumerate}
					\item
						The tutorial screen shall display high-level tutorial categories for the user to choose from
					\item
						% Tutorial drop-downs
						The tutorial categories shall display drop-downs of their contained tutorials upon being tapped
					\item
						The user shall be taken to the appropriate tutorial upon tapping a contained tutorial
						\begin{enumerate}
							\item
								The tutorial shall be a text-based activity with scrolling
							\item
								The user shall be able to navigate back to other menus from the tutorial
						\end{enumerate}
				\end{enumerate}
			\item
				The UI shall contain games
				\begin{enumerate}
					\item
						The UI shall provide a way for the user to select from the application's games
						\begin{enumerate}
							\item
								The game select screen shall display categories for the user to chose from
							\item
								The user shall be taken to the game options screen upon tapping a catagory
						\end{enumerate}
					\item
						The UI shall provide a way for user to configure the game parameters
						\begin{enumerate}
							\item
								The game options screen shall display options for the user to manipulate game time duration
							\item
								The game options screen shall display options for the user to specify the number of questions
							\item
								The game options screen shall display options for the user to specify whether or not problems will be scored
							\item
								The user shall be taken to the appropriate game upon tapping a game start button
						\end{enumerate}
					\item
						The UI shall provide a way for the user to play the application's games
						\begin{enumerate}
							\item
								The in-game screens shall contain a button to end the current game
						\end{enumerate}
				\end{enumerate}
			\item
				% User profile
				The UI shall contain a section for viewing the user's statistics
				\begin{enumerate}
					\item
						The statistics screen shall display high-level statistics categories for the user to view
						\begin{enumerate}
							\item
								The high-level categories shall be displayed in a scrollable, horizontal bar
						\end{enumerate}
					\item
						The information within each category shall be presented graphically
				\end{enumerate}
		\end{enumerate}
	\item
		Tutorial
		\begin{enumerate}
			\item
				Each tutorial shall be comprised of five attributes: name, category, difficulty, content, completion
				\begin{enumerate}
					\item
						A tutorial's name shall describe the material to be learned in the content
					\item
						A tutorial's category shall describe  which mathematical subject the content falls under
					\item
						A tutorial's difficulty shall display the level of difficulty of the tutorial
						\begin{enumerate}
							\item
								A tutorial's difficulty is relative to that of another
						\end{enumerate}
					\item
						A tutorial's content shall be comprised of learning material and example problems
						\begin{enumerate}
							\item
								Learning material shall be described in steps
							\item
								Example problems shall be worked out and solved within the content
						\end{enumerate}
					\item
						A tutorial's completion shall display how much of a tutorial has been completed
				\end{enumerate}
			\item
				All tutorials shall be sorted by name, category, and difficulty
				\begin{enumerate}
					\item
						Tutorials shall be sorted by category
					\item
						Tutorials of the same category shall then be sorted by difficulty
					\item
						Tutorials of the same difficulty shall then be sorted by name
				\end{enumerate}
		\end{enumerate}
	\item
		User profile
	\item
		Feedback
\end{enumerate}
%===============================================================================
\end{document}

